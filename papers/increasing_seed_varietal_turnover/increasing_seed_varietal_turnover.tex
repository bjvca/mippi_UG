\batchmode
\makeatletter
\def\input@path{{/home/bjvca/data/projects/OneCG/MIPP/papers/increasing_seed_varietal_turnover/}}
\makeatother
\documentclass[english]{article}\usepackage[]{graphicx}\usepackage[]{xcolor}
% maxwidth is the original width if it is less than linewidth
% otherwise use linewidth (to make sure the graphics do not exceed the margin)
\makeatletter
\def\maxwidth{ %
  \ifdim\Gin@nat@width>\linewidth
    \linewidth
  \else
    \Gin@nat@width
  \fi
}
\makeatother

\definecolor{fgcolor}{rgb}{0.345, 0.345, 0.345}
\newcommand{\hlnum}[1]{\textcolor[rgb]{0.686,0.059,0.569}{#1}}%
\newcommand{\hlstr}[1]{\textcolor[rgb]{0.192,0.494,0.8}{#1}}%
\newcommand{\hlcom}[1]{\textcolor[rgb]{0.678,0.584,0.686}{\textit{#1}}}%
\newcommand{\hlopt}[1]{\textcolor[rgb]{0,0,0}{#1}}%
\newcommand{\hlstd}[1]{\textcolor[rgb]{0.345,0.345,0.345}{#1}}%
\newcommand{\hlkwa}[1]{\textcolor[rgb]{0.161,0.373,0.58}{\textbf{#1}}}%
\newcommand{\hlkwb}[1]{\textcolor[rgb]{0.69,0.353,0.396}{#1}}%
\newcommand{\hlkwc}[1]{\textcolor[rgb]{0.333,0.667,0.333}{#1}}%
\newcommand{\hlkwd}[1]{\textcolor[rgb]{0.737,0.353,0.396}{\textbf{#1}}}%
\let\hlipl\hlkwb

\usepackage{framed}
\makeatletter
\newenvironment{kframe}{%
 \def\at@end@of@kframe{}%
 \ifinner\ifhmode%
  \def\at@end@of@kframe{\end{minipage}}%
  \begin{minipage}{\columnwidth}%
 \fi\fi%
 \def\FrameCommand##1{\hskip\@totalleftmargin \hskip-\fboxsep
 \colorbox{shadecolor}{##1}\hskip-\fboxsep
     % There is no \\@totalrightmargin, so:
     \hskip-\linewidth \hskip-\@totalleftmargin \hskip\columnwidth}%
 \MakeFramed {\advance\hsize-\width
   \@totalleftmargin\z@ \linewidth\hsize
   \@setminipage}}%
 {\par\unskip\endMakeFramed%
 \at@end@of@kframe}
\makeatother

\definecolor{shadecolor}{rgb}{.97, .97, .97}
\definecolor{messagecolor}{rgb}{0, 0, 0}
\definecolor{warningcolor}{rgb}{1, 0, 1}
\definecolor{errorcolor}{rgb}{1, 0, 0}
\newenvironment{knitrout}{}{} % an empty environment to be redefined in TeX

\usepackage{alltt}
\usepackage[T1]{fontenc}
\usepackage[utf8]{inputenc}
\usepackage{color}
\usepackage{babel}
\usepackage{graphicx}
\usepackage{rotfloat}
\usepackage[authoryear]{natbib}
\usepackage[unicode=true,pdfusetitle,
 bookmarks=true,bookmarksnumbered=false,bookmarksopen=false,
 breaklinks=true,pdfborder={0 0 0},pdfborderstyle={},backref=false,colorlinks=true]
 {hyperref}
\hypersetup{
 linkcolor=black, urlcolor=blue, citecolor=blue, pdfstartview={FitH}, hyperfootnotes=false, unicode=true}
\IfFileExists{upquote.sty}{\usepackage{upquote}}{}
\begin{document}
\title{\author{Bjorn Van Campenhout\thanks{Innovation Policy and Scaling Unit, International Food Policy Research Institute, Leuven, Belgium}, Leocardia Nabwire\thanks{Innovation Policy and Scaling Unit, International Food Policy Research Institute, Kampala, Uganda}, Berber Kramer\thanks{Markets Trade and Institutions Division, International Food Policy Research Institute, Nairobi, Kenya}, \\ Carly Trachtman\thanks{Markets Trade and Institutions Division, International Food Policy Research Institute, Washington DC, USA}, Gashaw Tadesse Abate\footnotemark[4]}Increasing
Adoption and Varietal Turnover of Seed---The Role of Producer and
Consumer traits}
\maketitle
\begin{abstract}
This study evaluates the effectiveness of two interventions aimed
at increasing the adoption of improved maize seed varieties among
smallholder farmers in eastern Uganda. The first intervention involves
providing farmers with free seed trial packs, allowing them to directly
experience the agronomic benefits, such as higher yield potential
and drought resistance. The second intervention consists of organizing
cooking demonstrations and blind tasting sessions to compare maize
flour from the improved variety with local varieties, focusing on
consumption traits like palatability, texture, and ease of cooking.
Our findings indicate that the seed trial packs significantly enhance
farmers' perceptions of the seed's production traits, while the cooking
demonstrations improve appreciation for its consumption attributes.
There is evidence suggesting that the cooking demonstration intervention
increases the adoption of improved seed varieties. However, farmers
who received the seed trial packs are more likely to reuse the harvested
grain as seed in the subsequent season. We conclude that although
the seed trial pack intervention has the greater overall impact, addressing
additional constraints is necessary to achieve higher seed varietal
turnover.

Keywords: technology adoption, subsidies, demonstration.

JEL: Q16, H24, O33, D91
\end{abstract}

\section{Motivation}

To sustainably feed a growing population without exacerbating climate
change and harming biodiversity, more food must be grown on less land.
Green revolution technologies, particularly improved planting materials,
play a crucial role as they promise higher yields and better adaptation
to challenging environmental conditions such as drought. Historically,
breeders have focused primarily on production-related traits like
yield, drought tolerance, and pest resistance. Currently, there is
an increasing emphasis on consumer-oriented attributes, including
taste, color, texture, and ease of cooking. For these new technologies
to be widely adopted, it is essential to raise farmers' awareness
of these beneficial traits.

Similar to the historical focus in breeding, sensitization efforts
often emphasize production traits. A common strategy to encourage
farmers to learn about the production traits of a new seed variety
is through seed trial packs.\footnote{One may argue that seed trial packs also enable farmers to discover
consumer traits after they harvest the grain obtained from the seed
trial pack. However, this would require farmers to keep the grain
separate from grain obtained from local seed.} These packs, typically containing small quantities of seed (e.g.,
1 kg), are provided for free, allowing farmers to test them on a small
portion of their plot. The goal is to encourage farmers to invest
in the seed in subsequent seasons if they find it beneficial.

From a theoretical standpoint, seed trial packs can be viewed as a
form of subsidy. Subsidies are often employed to promote adoption
and are justified for several reasons. First, resource-constrained
farmers may perceive the risk premium of a new product as too high
to justify investment. More broadly, even without considering risk
premia, farmers may lack the resources to purchase the seed, especially
if complementary inputs and effort are required \citep[see][]{miehe2023miracle}.
A one-time subsidy can help initiate adoption, as supported by the
theory of micro poverty traps, which posits that households and individuals
remain in chronic poverty because they cannot self-finance investments
necessary for high returns \citep{barrett2006fractal}. Second, it
is well known that farmers learn form others when new a technology
is introduced \citep{10.1257/aer.100.1.35,campenhout2021role} In
case where there are positive externalizes, subsidies may be justified.
Finally, the government may want to inject new seed into the existing
degenerated seed stock to increase overall vigor of seed circulating
in society \citep{mcguire2016seed}.

Surprisingly few studies directly evaluate the effectiveness of seed
trial packs in accelerating technology adoption. \citet{biedny2020demonstration}
found that in Tanzania, adding trial packs to demonstration plots
within the framework of village-based agricultural advisors did not
significantly impact input sales, orders received, or learning outcomes.
In many studies, the focus is not on the seed packs themselves but
on some attribute of the seed, such as its risk reduction potential
\citep[eg.][]{boucher2021bundling}. \citet{MORGAN2020101955} examined
different extension approaches, including the use of trial packs,
but their focus was on farmers' willingness to pay, elicited through
a Becker-DeGroot-Marschak (BDM) auction, rather than subsequent adoption
of the new technology. They found that in the southern highlands of
Tanzania, bean farmers' willingness to pay was not influenced by seed
trial packs. Most studies on seed trial packs are limited to a single
season and primarily investigate the immediate use of the seed trial
pack. Subsequent adoption is often assessed by asking farmers about
their intention to use the seed in the next season, which can be prone
to social desirability bias. 

It is much harder to find examples of strategies that emphasize consumption-related
traits. This is surprising, considering that in many contexts where
adoption is lacking, farmers consume most of the crops they grow.
Exploratory data analysis indicates that both ease of cooking and
taste are important characteristics influencing farmers' choice of
which varieties to adopt. Interventions that focus on demonstrating
consumption traits, such as cooking demonstrations and tasting sessions,
are typically concerned with nutrition education and rarely examine
subsequent adoption in the following season \citep[eg.][]{REICKS2014259}.

In this paper, the focus is on increasing seed varietal turnover by
highlighting both the production and consumption attributes of improved
maize seed varieties to farmers in their roles as producers and consumers.
To achieve this, a field experiment is conducted to test the relative
effectiveness of two interventions using a cluster randomized control
trial with a simple factorial design. The first intervention targets
production traits by providing farmers with 1 kg improved seed trial
packs.\footnote{We use the term ``improved variety'' throughout this paper to refer
to both maize hybrids and open pollinated varieties marketed and sold
in our study areas, as opposed to farmer-saved seed or seed obtained
through farmer-to-farmer exchanges which, in the specific context
of maize, may be less effective due to cross pollination and genetic
drift over multiple generations, or due to poor seed storage and handling
between seasons.} The seed used is a hybrid variety called Bazooka, which, despite
being widely available, is relatively new and thus not widely adopted
by farmers. The primary production trait of this seed is its potential
for increased yield. Orthogonal to this intervention, a second was
introduced that involves cooking demonstrations and tasting sessions.
These demonstrations allow farmers to familiarize themselves with
consumption attributes of maize flour derived from the grain grown
using improved seed varieties. Additionally, farmers are provided
with maize flour samples to further explore consumption traits at
home. 

Somewhat surprisingly, we find that farmers who received a free seed
sample are actually less likely to use improved seed in the next season.
Further analysis suggests this is because farmers who received Bazooka
for free used grain obtained from this seed as planting material in
the next season. Since we categorize recycled hybrid seed as local
seed, these farmers are considered non-adopters in our analysis. However,
we do find that the cooking demonstration and tasting session increased
the use of fresh Bazooka, with some indications that this also led
to higher maize productivity.

While we expected that increased attention to consumption traits may
affect the gender balance in decision-making with respect to seed
choice, we do not find this effect. We also test if making consumption
traits more salient would increase consumption, but also here we do
not find an effect. We also do not find effects further down the impact
pathway in terms of food security or welfare.

Exploring some of the impact pathways, we do find that the cooking
demonstration and tasting session increased the share of farmers that
rank improved seed higher on a range of consumption attributes such
as taste, portions, appearance, and ease of cooking. Also in line
with the theory of change, we find that farmers that received the
seed trial pack rank improved seed higher in terms of production-related
attributes such as yield, abiotic and biotic stress resistance, time
to maturity, and germination rate. Interestingly, the seed trial pack
also positively affects how farmers think about consumption traits
of improved seed, suggesting farmers consume grain obtained from the
seed trial pack and pay close attention to consumption attributes.

The rest of the paper is organized as follows. The next section presents
research methods with sub-sections for the experimental layout, the
treatments, estimation and inference, and timeline. We then turn to
the data and discuss the study context, present balance tables, and
look at attrition. The next section presents the findings, starting
with adoption. We further look at how the interventions affect decision-making,
food security, and well-being. We also study some of the potential
underlying mechanisms. A final section concludes.

\section{Methods }

\subsection{Experimental design}

In this study, we employ a field experiment to evaluate the effectiveness
of two interventions, which are elaborated in the subsequent section.
This evaluation is conducted using a cluster randomized controlled
trial (RCT) structured as a 2x2 factorial design. Each factor includes
both a control and a treatment level, with interventions clustered
at the village level. Within each village, a predetermined number
of households were randomly selected.

The first factor pertains to the seed trial pack treatment. Farmers
in the treatment villages receive a complimentary sample of a new,
improved seed variety (bazooka), whereas those in the control villages
do not receive this free sample pack.\footnote{They do get something of similar value---a so-called token of appreciation---to
account for potential income effects} 

The second factor involves a cooking demonstration and tasting session.
In the treatment villages, farmers participate in a demonstration
where they can taste food prepared with the promoted seed variety
and compare it directly to food made with the local variety. Additionally,
all participants receive a free sample of maize flour derived from
the improved seed variety to try at home. In the control villages,
these events are not organized.

Treatment assignment is conducted at the village level to ensure that
control and treatment conditions are not confounded by proximity.
Specifically, this approach prevents a situation where a control farmer,
who receives only a bar of soap as a token of gratitude, lives adjacent
to a treatment farmer who receives a free bag of maize seed. By assigning
treatments at the village level, we also mitigate potential concerns
about spillover effects, where benefits or changes experienced by
treatment farmers might unintentionally influence the control farmers.
This strategy helps to maintain the integrity of the experimental
conditions and ensures that the observed effects can be more accurately
attributed to the interventions being tested. 

\subsection{Treatments}

The first intervention is straightforward and involves providing a
seed trial pack to the household member responsible for most maize
cultivation decisions. This trial pack contains an improved seed variety
(hybrid seed) that is currently available in the market but has not
yet been widely adopted by farmers. Specifically, we used 1 kg bags
of bazooka seed, which is sufficient to plant approximately 1/8 of
an acre. The control condition for this factor is the absence of a
seed trial pack, meaning these households do not receive the pack.
However, both the treatment and control groups are informed about
the existence and benefits of the improved seed variety. This allows
us to isolate the effect of the trial pack from the effects of merely
having knowledge about the improved seeds.

For the second factor, the treatment level consists of a cooking demonstration
and tasting event. Here, sampled farmers of the treatment villages
are invited to a central place (the village chairperson's residence)
for a facilitated meeting. The facilitator starts by asking the group
to mention the most commonly grown varieties by farmers in the village.
These varieties are then grouped into ``improved seed varieties''
and ``local seed varieties'' on a flip-chart.\footnote{The terms to the seed types in the local language were ``Dhuuma Omulongosemu''
for ``improved seed'' and ``Dhuuma Omusoga'' for local seed. The
latter is clearly derived from the name of the region, Busoga.} Farmers are then asked to rate the two categories based on various
consumption attributes by a show of hands. To facilitate the discussion,
flip charts display the five most common consumption traits: taste,
texture, color, aroma, and the degree to which the flour expands during
cooking.\footnote{These consumption traits were identified through focus group discussions.
One notable trait that emerged during these discussions, which was
integral to designing the treatment, is the extent to which maize
flour expands during cooking, yielding \textquotedbl more food from
less flour.\textquotedbl{} This characteristic is linked to the starch
content of maize flour, which is a carbohydrate and a natural component
of most plants, including fruits, vegetables, and grains. When starch
is heated in water, the granules swell and burst, releasing glucose
molecules into the water. This expansion property of maize flour was
particularly valued by women.} Farmers can add as many traits as they see fit.

After the rating, we proceed with a blind tasting. The facilitator
asks a volunteer from the participants to prepare posho, a thick,
dense porridge made by mixing maize flour with boiling water until
it reaches a dough-like consistency. Two meals are prepared: one using
flour obtained from local seed and one using flour derived from Bazooka
seed (the hybrid seed variety used in the seed trial pack). The cook
is unaware of which flour corresponds to which maize type.\footnote{The research team provided all necessary utensils for the session,
including a gas stove, pots, aprons, and even a chef's hat. We ensured
that the two dishes differed only in terms of the flour used by employing
the same cook and starting with identical amounts of flour, measured
on a weighing scale. Additionally, the order in which the meals were
cooked was randomized across sessions.} The resulting dishes are then displayed on a table, and farmers are
invited to taste the two varieties, labeled as the variety on the
left and the variety on the right. The farmers then rate the two varieties
on various consumption attributes, and they are asked to indicate
which of the two samples is superior for each attribute by a show
of hands.

Finally, the results are discussed within the group. Participants
are informed that one of the two samples was made from flour obtained
from local maize, while the other was from an improved maize type
called Bazooka. The facilitator asks the farmers to guess which sample
was based on flour from the local variety and which was from the improved
variety. After gathering the guesses, the facilitator reveals the
correct answers.\footnote{During field testing, a large majority of farmers initially stated
that the local seed excelled in nearly all dimensions, such as having
a sweeter taste, whiter appearance, and better aroma. However, during
the tasting, almost all farmers consistently rated the sample made
from Bazooka maize as superior. After the tasting, most farmers incorrectly
identified the superior sample as being from the local variety, when
in fact, it was made from Bazooka maize.}

\subsection{Estimation and inference\label{subsec:Estimation-and-inference}}

We will use Analysis of Covariance (ANCOVA) models to assess the impact
of the interventions, focusing specifically on the Intent-to-Treat
effect. Given that randomization was conducted at the village level,
we will estimate the following equation:

\begin{equation}
Y_{ij}=\alpha+\beta_{T}T_{j}^{T}+\beta_{D}T_{j}^{D}+\beta_{I}T_{j}^{T}T_{j}^{D}+\delta Y_{ij}^{B}+\varepsilon_{ij}\label{eq:one}
\end{equation}

where $T_{j}^{T}$ is an indicator variable that is one if village
\emph{j} was randomly assigned to the seed trial pack intervention
(and zero otherwise), and $T_{j}^{D}$ is an indicator variable that
is one if village \emph{j} was randomly assigned to the cooking demonstration
and blind tasting intervention (and zero otherwise). Outcomes are
measured at the individual level $\left(Y_{ij}\right)$. We also allow
for an interaction effect between the two interventions and control
for baseline outcomes $\left(Y_{ij}^{B}\right)$ to improve precision.
We will apply a cluster-robust variance estimator with the \textquotedbl bias-reduced
linearization\textquotedbl{} small-sample correction \citep{imbens2016robust}.
Standard errors are clustered at the level of randomization (village
level). 

Factorial designs are commonly employed to evaluate multiple treatments
within a single experiment. However, combining treatment cells to
enhance statistical power can lead to unintended consequences if there
is an interaction between the treatments \citep{10.1162/rest_a_01317}.
While our design is powered for a complete set of interactions (as
in equation \ref{eq:one}), we may wish to enhance statistical power
by pooling observations across the orthogonal treatments if we find
that a treatment effect is smaller than the minimal detectable effect
size assumed during power calculations. To achieve this, we will treat
the orthogonal treatment as a covariate to adjust for, and interact
the treatment variable with the demeaned orthogonal treatment. This
approach provides a more robust and unbiased estimate of the treatment
effect:

\begin{equation}
Y_{ij}=\alpha+\beta_{M}T_{i}^{M}+\beta_{O}\left(T_{i}^{O}-\bar{T^{O}}\right)+\beta_{I}T_{i}^{M}\left(T_{i}^{O}-\bar{T^{O}}\right)+\delta Y_{0ij}^{B}+\varepsilon_{ij}\label{eq:pooled}
\end{equation}

Where now $T_{i}^{M}$ is a dummy for the main treatment (the seed
trial pack or the cooking demonstration respectively) and $T_{i}^{O}$
is a dummy for the orthogonal treatment (which enters in deviations
from its means).

Since we will evaluate treatment effects across a range of outcome
measures, it is essential to address the multiple comparisons problem.
Following the approach outlined by \citet{doi:10.1198/016214508000000841},
we aggregate various outcome measures within a given family into summary
indices. Each index is calculated as a weighted mean of the standardized
values of the outcome variables. The weights for this efficient generalized
least squares estimator are determined to maximize the information
captured by the index, giving less weight to outcomes that are highly
correlated with one another. This method of combining outcomes into
indices is a widely adopted strategy to mitigate the risk of over-rejecting
the null hypothesis due to multiple comparisons.

\subsection{Timeline}

To assess the impact of seed trial packs, two agricultural seasons
are required. Seed samples are distributed to farmers before the first
season, allowing them to experiment with the new seed in their fields.
The impact of the seed trial pack is evaluated in the subsequent season,
as farmers adjust their practices based on their experiences with
the seeds from the first season. In contrast, the cooking demonstration
and tasting session can result in immediate behavior changes following
the intervention.

\begin{figure}
\begin{centering}
\includegraphics[scale=0.35]{0_home_bjvca_data_projects_OneCG_MIPP_papers_increasing_seed_varietal_turnover_timeline.png}
\par\end{centering}
\caption{Timeline\label{fig:Timeline}}
\end{figure}

Fortunately, the study area experiences two maize growing seasons
each calendar year. The first season, locally known as Entoigo, runs
from March/April to June/July, while the second season, referred to
as Nsambya, extends from August/September to November/December. We
distributed the seed trial packs along with baseline data collection
a few months before planting commenced for the first agricultural
season in 2023. Following this, we conducted the cooking demonstration
and tasting session in time for the second season of 2023. Endline
data was collected in February 2024, after the conclusion of the second
season. This timeline is also depicted in Figure \ref{fig:Timeline}.

\section{Sample}

Sample size was determined through a series of power simulations detailed
in the pre-registered pre-analysis plan.\footnote{he primary outcome used in these simulations is a binary indicator
of improved seed use at the farmer level. Data from a previous project,
which included 3,450 smallholder maize farmers across 345 villages,
was used to estimate the intra-cluster (within-village) correlation
for this outcome \citep{miehe2023perceived}. We assumed a treatment
effect of a 13.5 percentage point increase for both the seed trial
treatment and the consumption treatment, and a 23.5 percentage point
increase for the interaction effect. The R code used for the simulations
is available \href{https://github.com/bjvca/mippi_UG/blob/master/study_design/power_analysis.R}{here}.
The pre-analysis plan can be accessed \href{https://github.com/bjvca/mippi_UG/blob/master/study_design/MIPPI_PAP.pdf}{here}.} The simulations resulted in a sample design of 148 villages, with
10 households in each village. In this setup, 74 villages (740 households)
will receive a free trial pack, and 74 villages (740 households) will
be exposed to the consumption treatment. Approximately half of these
villages will receive both treatments, resulting in around 37 villages
(370 households) receiving both interventions. With this design, we
are not powered to detect all three effects simultaneously. Specifically,
we can estimate a positive effect for both treatments and their interaction
at the 5\% significance level in only 68\% of cases. However, when
considering the treatments individually, we achieve conventional power
levels for both treatments and a power of up to 0.97 for the interaction
effect. We are confident in identifying at least one of the three
parameters of interest---seed packs, the consumer intervention, or
their interaction.

\subsection{Context}

The study was conducted in Eastern Uganda, specifically in the Busoga
Kingdom. We selected our sample from four districts---Bugiri, Iganga,
Kamuli, and Mayuge---chosen for their relatively low adoption rates
of the improved seed variety (Bazooka) and their well-established
network of agro-input dealers, which helps mitigate access issues.
These districts were identified based on data from a previous study
that matched these criteria.

The study population comprises smallholder maize farmers. To obtain
a representative random sample, villages were selected with probability
proportional to the number of households within each village. Within
each sampled village, 10 households were randomly chosen to participate
in the study.



\subsection{Balance test and descriptive statistics}

We pre-registered 10 variables to assess balance in our design during
baseline data collection. These variables were selected to offer a
comprehensive description of a representative farmer in our sample.
Half of the variables are characteristics that are unlikely to be
influenced by the intervention, while the other five are drawn from
the pre-registered primary and secondary endline outcomes.

Table \ref{balance_table} shows sample means in the first column
(and standard deviations below). We see that the average household
head in our sample was about 49
years old at the time of the baseline survey. About half of household
heads had finished primary education. In 20 percent of the households,
the household head was a woman. Households in the area are large,
with on average about 8 to 9 individuals. The average distance to
the nearest agro-input shop where maize seed of an improved variety
can be bought is about 4 kilometers.

We first inquired whether farmers had used “quality maize seed, such
as an Open Pollinating Variety (OPV) or hybrid seed, on any of their
plots during the previous season (Nsambya of 2022).” According to
the baseline data, approximately 40 percent of households affirmed
using such seeds. Subsequently, we posed a more specific question
regarding the exact type of seed used on a randomly selected plot.\footnote{See Section \ref{subsec:Adoption} for more information on the rationale
of using of a randomly selected plot.} The seed type of interest was Bazooka, the hybrid seed that is also
utilized in our experiment. At baseline, only about 7 percent of farmers
reported having used Bazooka seed on the randomly selected plot in
the previous season. We also asked where the seed used on the randomly
selected plot was obtained. Results indicate that approximately 29
percent of farmers sourced their seed from formal channels, such as
agro-input dealers, non-governmental organizations, or the government
extension system (Operation Wealth Creation). Conversely, 47 percent
of farmers reported reusing seed from previous seasons, with some
having used it for more than four seasons. Notably, Open Pollinated
Varieties that have been used more than four times tend to lose their
yield advantage. Finally, the average farmer harvested about 390 kilograms
per acre on the randomly selected plot at baseline. 

\begin{sidewaystable}
\caption{Baseline Balance\label{balance_table}}

\footnotesize
\begin{center}
\begin{tabular}{rcccccc}
\hline\hline
& & \multicolumn{2}{c}{\textit{Pooled model}} & \multicolumn{3}{c}{\textit{Interacted model}} \\
& mean & trial & cons & trial & cons & interact \\
\hline
\\
Age of HH head (in years)
& 48.62
& -0.31$^{}$
& -0.65$^{}$ 
& -1.53$^{}$ 
& -1.87$^{}$
& 2.45$^{}$ \\

& (13.58) & (0.86)  & (0.86) & (1.33)  & (1.23)  & (1.72)
 \\
HH has finished primary education?
& 0.51
& 0.02$^{}$
& 0.00$^{}$ 
& 0.01$^{}$ 
& -0.02$^{}$
& 0.03$^{}$ \\

& (0.50) & (0.03)  & (0.03) & (0.05)  & (0.04)  & (0.06)
 \\
Gender of household head (1=male)
& 0.80
& -0.03$^{}$
& -0.04$^{+}$ 
& -0.03$^{}$ 
& -0.04$^{}$
& 0.00$^{}$ \\

& (0.40) & (0.03)  & (0.03) & (0.03)  & (0.03)  & (0.05)
 \\

Household size
& 8.24
& -0.08$^{}$
& -0.03$^{}$ 
& -0.36$^{}$ 
& -0.31$^{}$
& 0.56$^{}$ \\

& (3.91) & (0.26)  & (0.26) & (0.36)  & (0.35)  & (0.52)
 \\

Distance of homestead to nearest agro-input shop (km)
& 4.08
& -0.21$^{}$
& 0.07$^{}$ 
& -0.11$^{}$ 
& 0.16$^{}$
& -0.18$^{}$ \\

& (3.54) & (0.38)  & (0.38) & (0.57)  & (0.62)  & (0.76)
 \\
Has used quality maize seed on any plot in last season?
& 0.40
& -0.04$^{}$
& -0.02$^{}$ 
& -0.03$^{}$ 
& -0.01$^{}$
& -0.03$^{}$ \\

& (0.49) & (0.03)  & (0.03) & (0.05)  & (0.04)  & (0.07)
 \\

Has used the promoted seed (bazooka) on a randomly chosen plot in the last season?
& 0.07
& -0.01$^{}$
& -0.01$^{}$ 
& 0.02$^{}$ 
& 0.02$^{}$
& -0.05$^{}$ \\

& (0.26) & (0.02)  & (0.02) & (0.02)  & (0.03)  & (0.04)
 \\

Seed on random plot was obtained from formal seed source? 
& 0.29
& -0.04$^{}$
& 0.02$^{}$ 
& -0.01$^{}$ 
& 0.05$^{}$
& -0.06$^{}$ \\

& (0.46) & (0.03)  & (0.03) & (0.04)  & (0.04)  & (0.06)
 \\
Used seed that is recycled more than 5 seasons on randomly selected plot? 
& 0.47
& 0.03$^{}$
& 0.05$^{}$ 
& 0.03$^{}$ 
& 0.04$^{}$
& 0.01$^{}$ \\

& (0.50) & (0.03)  & (0.03) & (0.05)  & (0.05)  & (0.07)
 \\

Maize yields on a randomly chosen plot in last season (kg/acres)
& 390.79
& -0.08$^{}$
& -0.01$^{}$ 
& -0.05$^{}$ 
& 0.01$^{}$
& -0.05$^{}$ \\

& (309.05) & (0.06)  & (0.06) & (0.07)  & (0.07)  & (0.12)
 \\
\\
\hline\hline
\end{tabular}
\end{center}
\scriptsize
Note: Column (1) reports sample means at baseline and standard deviations below;  columns (2)-(3) report differences between treatment and control groups estimated using the pooled model of Equation  \ref{eq:pooled}; (4)-(5) report differences between treatment and control groups estimated using the fully interacted model of Equation \ref{eq:one}; column (6) is the interaction effect. Standared errors are below the estimates and are clustered at the level of randomization; **, *, and + denote significance at the 1, 5 and 10\% levels. Productivity is log transformed (means and standard deviations are in levels).
\end{sidewaystable}

The table also reports differences between the relevant treatment
and control groups, as estimated through the pooled model (Equation
\ref{eq:pooled}) as well as the fully interacted model (Equation
\ref{eq:one}). In particular, the second column in Table \ref{balance_table}
corresponds to $\beta_{M}$ in Equation \ref{eq:pooled} for the seed
trial treatment (in which case the cooking demonstration and tasting
treatment is considered the orthogonal treatment). In the third column,
this is reversed, showing $\beta_{M}$ in Equation \ref{eq:pooled}
for the cooking demonstration and tasting treatment (with the seed
trial treatment now considered the orthogonal treatment). The fourth,
fifth and sixth columns correspond to $\beta_{T}$, $\beta_{D}$ and
$\beta_{I}$ respectively in Equation \ref{eq:one}.

The estimates indicate good overall balance. Specifically, the proportion
of male household heads is about 4 percentage points lower in the
subsample assigned to the cooking demonstration treatment compared
to those not exposed to the intervention, but this difference is only
significant at the 10 percent level. Additionally, this group has
a slightly higher proportion of seed recyclers, with the difference
being significant at the 5 percent level. However, considering that
out of 50 comparisons, we would expect 2.5 to be significant at the
5 percent level and 5 to be significant at the 10 percent level by
chance alone, we conclude that the randomization appears to have been
effective.

\subsection{Attrition and compliance}

Attrition was minimal: we successfully located all but four of the
1,560 households interviewed at baseline. However, some of these households
could not be interviewed due to circumstances such as burial or illness,
or they refused participation, reducing the effective sample size
at endline to 1,538.

One of the first questions we asked was whether farmers recalled the
treatment, irrespective of their treatment group. This question serves
as a useful proxy to assess compliance. For the trial pack treatment,
98 percent of farmers in the treatment group indicated that they received
a seed trial pack from us in March 2023. However, 8.6 percent of farmers
in the control group also indicated that they received a seed pack
from us, representing excess coverage. For the cooking demonstration
treatment, 92 percent of farmers in the treatment group recalled being
invited to a cooking and tasting demonstration, indicating a failure
to reach rate of 8 percent. Meanwhile, 96 percent of farmers in the
control group did not recall a cooking and tasting demonstration,
leading to an excess coverage rate of 4 percent.

\section{Results}

\subsection{Adoption\label{subsec:Adoption}}

Adoption of improved seed varieties is the primary outcome of interest
in this study. Table \ref{adoptiom} looks at adoption in general,
while Table \ref{adoptiom-1} asks more detailed questions about a
seed use and practices on a particular (randomly selected) plot within
the household. As mentioned in Section \ref{subsec:Estimation-and-inference},
outcomes are combined in a summary index following \citet{doi:10.1198/016214508000000841}
as a first safeguard to over-rejection due to multiple hypothesis
testing.

The structure of this table, and all other tables below, is the same
as the one used in Table \ref{balance_table}. It reports overall
sample means at endline, as well as intent-to-treat estimates for
the pooled model in Equation \ref{eq:pooled} and the fully interacted
model in Table \ref{eq:one}.

\begin{sidewaystable}
\caption{Adoption\label{adoptiom}}

\footnotesize
\begin{center}
\begin{tabular}{rccccccc}
\hline\hline
& & \multicolumn{2}{c}{\textit{Pooled model}} & \multicolumn{3}{c}{\textit{Interacted model}} \\
& mean & trial & demo & trial & demo & interact & nobs\\
\hline
\\
Has used quality maize seed on any plot in last season?
& 0.25
& -0.12$^{**}$
& 0.04$^{+}$ 
& -0.14$^{**}$ 
& 0.03$^{}$
& 0.02$^{}$ & 1495 \\

& (0.43) & (0.03)  & (0.02) & (0.03)  & (0.03)  & (0.05)
 \\

Has used fresh Bazooka on any plot in last season in last season?
& 0.09
& 0.02$^{}$
& 0.05$^{**}$ 
& 0.02$^{}$ 
& 0.05$^{*}$
& -0.01$^{}$ & 1495 \\

& (0.28) & (0.02)  & (0.02) & (0.02)  & (0.02)  & (0.03)
 \\

Number of plots planted with improved seed
& 0.33
& -0.16$^{**}$
& 0.05$^{}$ 
& -0.18$^{**}$ 
& 0.03$^{}$
& 0.05$^{}$ & 1495 \\

& (0.64) & (0.04)  & (0.04) & (0.05)  & (0.05)  & (0.08)
 \\

Number of plots with improved seed as share of total number of plots
& 0.23
& -0.13$^{**}$
& 0.03$^{}$ 
& -0.13$^{**}$ 
& 0.03$^{}$
& 0.01$^{}$ & 1494 \\

& (0.41) & (0.03)  & (0.03) & (0.03)  & (0.03)  & (0.05)
 \\


Area planted with improved seed (acres)
& 0.34
& -0.16$^{**}$
& 0.04$^{}$ 
& -0.22$^{**}$ 
& -0.02$^{}$
& 0.12$^{}$ & 1495 \\

& (0.81) & (0.05)  & (0.05) & (0.06)  & (0.06)  & (0.10)
 \\

Area planted with improved seed as a share of total maize cultivation area
& 0.23
& -0.13$^{**}$
& 0.03$^{}$ 
& -0.14$^{**}$ 
& 0.03$^{}$
& 0.01$^{}$ & 1495 \\

& (0.41) & (0.03)  & (0.03) & (0.03)  & (0.03)  & (0.05)
 \\
\\
Index
& -0.06
& -0.18$^{**}$
& 0.08$^{}$ 
& -0.21$^{**}$ 
& 0.06$^{}$
& 0.05$^{}$ & 1494 \\

& (0.85) & (0.05)  & (0.05) & (0.06)  & (0.06)  & (0.11)
 \\

\\


\hline\hline
\end{tabular}
\end{center}
\scriptsize
Note: Column (1) reports sample means at baseline and standard deviations below;  columns (2)-(3) report differences between treatment and control groups estimated using the pooled model of Equation  \ref{eq:pooled}; (4)-(5) report differences between treatment and control groups estimated using the fully interacted model of Equation \ref{eq:one}; column (6) is the interaction effect. Standared errors are below the estimates and are clustered at the level of randomization; **, *, and + denote significance at the 1, 5 and 10\% levels.
\end{sidewaystable}

This discrepancy is due to a change in the question used to measure
seed use. While at baseline we asked a single, broad question, at
endline we inquired about seed use on each plot individually. This
detailed information is then used to determine if quality maize seed
was used on at least one plot. At baseline, we left it to the farmer
to decide what quality seed means; at endline, we defined quality
seed as fresh hybrid seed or an OPV recycled a maximum of three times
from a trusted source.

We find that the seed trial pack intervention reduced the likelihood
of adoption, regardless of the model used. This reduction is substantial,
amounting to approximately half of the mean adoption rate. Conversely,
we do not observe any change in the use of improved seed as a result
of the cooking demonstration and tasting session.

We define a similar variable to examine the adoption of the specific
seed used in the experiment, coding this outcome as true only if the
seed is fresh and obtained from a trusted source. Overall, about 9
percent of farmers use fresh Bazooka seed on at least one plot. Notably,
when considering this outcome, the negative effect of the seed trial
pack disappears. In fact, we now find that farmer households exposed
to the cooking demonstration intervention are more likely to use fresh
Bazooka seed in the season following the intervention.

Next, using the same definition of improved seed as in the first question,
we determine the number of plots on which improved seed was planted.
However, it could also be that the intervention changes the total
number of plots used for cultivation, so we also scale the number
of plots on which improved seed was used by the total number of plots.
We find that improved seed was planted on 0.33 plots in the entire
sample. The seed trial pack reduced the number of plots on which improved
seed was planted. We further find that the seed trial pack intervention
also reduced the share of plots on which improved seed was planted
by 13 to 14 percentage points. This suggests that total number of
plots did not change, but that treated households dis-adopted.

A related outcome is the area planted with improved seed. We find
that, on average, households planted about 0.34 acres with improved
seed. However, in households that received a seed trial pack, the
area planted with improved seed was 0.16 acres less compared to households
that did not receive a seed trial pack. To account for potential changes
in total acreage due to the intervention, we also scale the area planted
with improved seed by the total area under cultivation. We observe
a reduction in the proportion of the total area planted with improved
seed, suggesting that the intervention led farmers to switch from
planting improved seed to planting local seed. Additionally, the summary
index confirms a significant and negative impact of the seed trial
pack on these adoption outcomes.

We now turn to plot-level outcomes for a more detailed analysis. Instead
of collecting data on all plots, we randomly selected one plot per
household on which we asked a range of detailed questions. These questions
covered not only seed use but also complementary input use and crop
management practices. Additionally, we use this data to extrapolate
yield estimates.\footnote{The decision to collect detailed data on only one randomly selected
plot per household, rather than on all plots, was based on the generally
high correlation between technology use, input use, and management
practices within a household. By focusing on a single plot, we aimed
to increase the number of farmers included in the sample, thereby
enhancing statistical power. While one might be concerned that farmers
could use different seeds on different plots, potentially leading
to misclassification of adopters if the \textquotedbl wrong\textquotedbl{}
plot is chosen, our data indicates that this is not a significant
issue. Adoption outcomes measured at the plot level are virtually
identical to those measured across any plot within the household.} 

Table \ref{adoptiom-1} confirms results from Table \ref{adoptiom}
with negative effects on the adoption of seed of an improved variety
for the seed trial pack treatment, but a positive effect on the use
of fresh bazooka seed from the cooking demonstration. The table further
indicates that reduced adoption leads to lower quantities of improved
seed used, both in absolute terms and when measured per acre.

Interestingly, we do not see that the reduced use of improved seed
had a negative impact on maize production or productivity. In fact,
there are some signs that the cooking demonstration and tasting sessions
actually increased production and productivity. In particular, we
find a 13 percent increase in maize productivity (as we use log transformations
in the regression). 

\begin{sidewaystable}
\caption{Adoption on random plot\label{adoptiom-1}}

\footnotesize
\begin{center}
\begin{tabular}{rccccccc}
\hline\hline
& & \multicolumn{2}{c}{\textit{Pooled model}} & \multicolumn{3}{c}{\textit{Interacted model}} \\
& mean & trial & demo & trial & demo & interact & nobs \\
\hline
\\
Has used quality maize seed on randomly selected plot in last season (yes=1)
& 0.23
& -0.12$^{**}$
& 0.03$^{}$ 
& -0.13$^{**}$ 
& 0.02$^{}$
& 0.02$^{}$ & 1495 \\

& (0.42) & (0.02)  & (0.02) & (0.03)  & (0.03)  & (0.05)
 \\

Has used fresh Bazooka on randomly selected plot in last season (yes=1)
& 0.07
& 0.01$^{}$
& 0.04$^{*}$ 
& 0.02$^{}$ 
& 0.05$^{**}$
& -0.02$^{}$ & 1495 \\

& (0.26) & (0.01)  & (0.01) & (0.02)  & (0.02)  & (0.03)
 \\

Quantity of improved seed used on randomly selected plot (kg)
& 1.33
& -0.89$^{**}$
& -0.13$^{}$ 
& -1.27$^{**}$ 
& -0.51$^{+}$
& 0.77$^{*}$ & 1438 \\

& (3.18) & (0.18)  & (0.17) & (0.25)  & (0.28)  & (0.35)
 \\

Quantity of improved seed used on randomly selected plot (kg/acre)
& 1.32
& -0.78$^{**}$
& -0.05$^{}$ 
& -1.12$^{**}$ 
& -0.39$^{}$
& 0.67$^{*}$ & 1411 \\

& (2.83) & (0.17)  & (0.17) & (0.23)  & (0.24)  & (0.34)
 \\

Maize production
& 331.20
& -0.07$^{}$
& 0.11$^{+}$ 
& -0.09$^{}$ 
& 0.09$^{}$
& 0.04$^{}$ & 1410 \\

& (340.92) & (0.06)  & (0.06) & (0.08)  & (0.09)  & (0.12)
 \\

Maize productivity
& 383.50
& 0.01$^{}$
& 0.13$^{*}$ 
& -0.01$^{}$ 
& 0.11$^{}$
& 0.04$^{}$ & 1375 \\

& (292.76) & (0.05)  & (0.05) & (0.08)  & (0.07)  & (0)
 \\

\\
Index
& -0.07
& -0.09$^{*}$
& 0.02$^{}$ 
& -0.16$^{**}$ 
& -0.05$^{}$
& 0.13$^{+}$ & 1331 \\

& (0.62) & (0.04)  & (0.04) & (0.05)  & (0.05)  & (0.07)
 \\
\\
\hline\hline
\end{tabular}
\end{center}
\scriptsize
Note: Column (1) reports sample means at baseline and standard deviations below;  columns (2)-(3) report differences between treatment and control groups estimated using the pooled model of Equation  \ref{eq:pooled}; (4)-(5) report differences between treatment and control groups estimated using the fully interacted model of Equation \ref{eq:one}; column (6) is the interaction effect. Standared errors are below the estimates and are clustered at the level of randomization; **, *, and + denote significance at the 1, 5 and 10\% levels. Production and productivity are log transformed (means and standard deviations are in levels).
\end{sidewaystable}

At first glance, the results for the seed trial packs seem counterintuitive.
The primary objective of distributing seed trial packs is to increase
the adoption of improved seed varieties in subsequent years. However,
our findings suggest that a significant portion of farmers in our
sample opted to recycle seed from the trial packs rather than purchasing
fresh Bazooka seed from agro-input dealers. According to our definition
of improved seed, recycled Bazooka seed does not qualify as such.
Furthermore, the intervention appears to have inadvertently crowded
out the use of fresh improved seed: the high yields experienced with
the trial packs led farmers to continue using recycled Bazooka seed.
In contrast, farmers in the control group may have been dissatisfied
with the local seed they were using. This dissatisfaction could have
led them to seek fresh seed or use an Open Pollinated Variety (OPV)
that was recycled fewer than four times.

\subsection{Decision Making and Disposal}

We hypothesize that highlighting the consumption traits of improved
seed varieties may shift the decision-making role concerning seed
selection and the use of harvested maize, potentially increasing the
involvement of female co-heads in these decisions. In Uganda, as in
many agriculture-based countries with pronounced gender norms, men
typically handle production and marketing decisions, while women are
more often involved in domestic responsibilities such as cooking.
The cooking demonstration and tasting session might increase men's
awareness that maize involves more than just yield, potentially creating
opportunities for women to participate more in decision-making processes.
Consequently, enhancing women's involvement in decision-making is
a significant focus of this study.

To explore this, we asked about decision-making within households.
Specifically, we inquired who decided which seed to use on the randomly
selected plot. We provided a range of response options and combined
these into an indicator variable signifying women involvement, which
is true if the decision was made solely by the woman, jointly by the
woman and the husband, or if the husband made the decision after consulting
the woman (as opposed to the husband deciding unilaterally). A similar
question was posed regarding decisions about the use of the harvest. 

\begin{sidewaystable}
\caption{Impact on women co-head involvement (random plot)\label{decision}}

\footnotesize
\begin{center}
\begin{tabular}{rccccccc}
\hline\hline
& & \multicolumn{2}{c}{\textit{Pooled model}} & \multicolumn{3}{c}{\textit{Interacted model}} \\
& mean & trial & cons & trial & cons & interact & nobs \\
\hline
\\
Woman involved in decision what to plant?
& 0.47
& 0.04$^{}$
& 0.02$^{}$ 
& 0.09$^{+}$ 
& 0.06$^{}$
& -0.09$^{}$ & 1469 \\

& (0.50) & (0.03)  & (0.03) & (0.04)  & (0.05)  & (0.06)
 \\

Women involved in what to do with harvest?
& 0.46
& 0.04$^{}$
& -0.01$^{}$ 
& 0.09$^{+}$ 
& 0.04$^{}$
& -0.10$^{}$ & 1469 \\

& (0.50) & (0.03)  & (0.03) & (0.05)  & (0.05)  & (0.07)
 \\
\\
Index
& 0.09
& 0.08$^{}$
& 0.01$^{}$ 
& 0.18$^{+}$ 
& 0.11$^{}$
& -0.20$^{}$ & 1469 \\

& (0.98) & (0.06)  & (0.06) & (0.09)  & (0.09)  & (0.13)
 \\

\\
\hline\hline
\end{tabular}
\end{center}
\scriptsize
Note: Column (1) reports sample means at baseline and standard deviations below;  columns (2)-(3) report differences between treatment and control groups estimated using the pooled model of Equation  \ref{eq:pooled}; (4)-(5) report differences between treatment and control groups estimated using the fully interacted model of Equation \ref{eq:one}; column (6) is the interaction effect. Standared errors are below the estimates and are clustered at the level of randomization; **, *, and + denote significance at the 1, 5 and 10\% levels.
\end{sidewaystable}

Table \ref{decision} reveals that approximately half of the women
are involved in decision-making related to seed use and harvest management.
Contrary to our expectations, the consumption intervention does not
show a significant effect on this involvement. However, there is a
positive effect associated with the seed trial pack intervention,
observed only in the fully interacted model, with significance at
the 10 percent level.

The interventions may also influence the use of maize harvested from
the selected plot. We first examined the proportion of the harvest
allocated to household consumption. We anticipated that the cooking
demonstration and tasting session would increase this share, based
on the premise that emphasizing consumption traits would lead to higher
consumption. Table \ref{disposal} shows that approximately three-quarters
of the maize is grown for home consumption. However, we did not find
a significant effect of either intervention on this outcome.

Maize that is not consumed is usually sold. We thus also look at what
share of the harvest is sold. Indeed, Table \ref{disposal} shows
that about 20 percent of seed is sold on average. We do not see any
effects of the interventions.

Finally, we examined the amount of maize grain reserved as planting
material for the next season. We anticipated a reduction in the quantity
of saved seed, given that both interventions aim to promote the adoption
of improved maize varieties and encourage varietal turnover. However,
we did not observe a significant effect of the interventions on this
outcome. This lack of effect may be attributed to the fact that, as
discussed in Section \ref{subsec:Adoption}, farmers appear to recycle
improved seeds also.

\begin{sidewaystable}
\caption{Disposal of harvest on random plot\label{disposal}}

\footnotesize
\begin{center}
\begin{tabular}{rccccccc}
\hline\hline
& & \multicolumn{2}{c}{\textit{Pooled model}} & \multicolumn{3}{c}{\textit{Interacted model}} \\
& mean & trial & cons & trial & cons & interact & nobs \\
\hline
\\
Share kept for consumption
& 74.24
& 0.63$^{}$
& -0.72$^{}$ 
& -1.21$^{}$ 
& -2.56$^{}$
& 3.69$^{}$ & 1449 \\

& (30.31) & (2.11)  & (2.10) & (3.00)  & (2.80)  & (4.21)
 \\

Share sold
& 20.52
& -0.74$^{}$
& -0.31$^{}$ 
& -1.50$^{}$ 
& -1.07$^{}$
& 1.53$^{}$ & 1495 \\

& (29.45) & (2.08)  & (2.08) & (2.93)  & (2.67)  & (4.16)
 \\

Maize kept as seed (kg)
& 4.37
& -0.83$^{}$
& -0.48$^{}$ 
& -0.24$^{}$ 
& 0.11$^{}$
& -1.19$^{}$ & 1321 \\

& (8.00) & (0.61)  & (0.61) & (0.86)  & (0.86)  & (1.21)
 \\
\\
Index
& 0.01
& 0.06$^{}$
& 0.03$^{}$ 
& 0.02$^{}$ 
& -0.01$^{}$
& 0.08$^{}$ & 1302 \\

& (0.72) & (0.05)  & (0.05) & (0.08)  & (0.08)  & (0.11)
 \\
\\
\hline\hline
\end{tabular}
\end{center}
\scriptsize
Note: Column (1) reports sample means at baseline and standard deviations below;  columns (2)-(3) report differences between treatment and control groups estimated using the pooled model of Equation  \ref{eq:pooled}; (4)-(5) report differences between treatment and control groups estimated using the fully interacted model of Equation \ref{eq:one}; column (6) is the interaction effect. Standared errors are below the estimates and are clustered at the level of randomization; **, *, and + denote significance at the 1, 5 and 10\% levels.
\end{sidewaystable}


\subsection{Well-being and food security}

Increasing the adoption of improved seed is not an end in itself;
ultimately, farmers seek to enhance their food security and overall
welfare. Consequently, Table \ref{welfare} examines the impact of
the interventions on various indicators of welfare and food security.

The first two questions assess subjective measures of relative well-being,
both within the village and over time. Approximately 35 percent of
farm households report that they perceive themselves as better off
compared to the average within their village. There is no significant
change in this perception following either the seed trial pack or
the cooking demonstration and tasting session. Additionally, 42 percent
of households believe they are better off than they were six months
ago---approximately one agricultural season prior. Notably, there
is some indication that farmers in the seed trial pack group are more
likely to feel better off compared to those who did not receive a
seed trial pack. This may be due to the perceived benefits from the
seed trial pack, either through improved harvests or a sense of improvement
even if they recycled the trial pack seed rather than adopting new
seed (despite yields not showing a significant increase, as detailed
in Table \ref{adoptiom-1}).

\begin{sidewaystable}
\caption{Welfare and food security\label{welfare}}

\footnotesize
\begin{center}
\begin{tabular}{rccccccc}
\hline\hline
& & \multicolumn{2}{c}{\textit{Pooled model}} & \multicolumn{3}{c}{\textit{Interacted model}} \\
& mean & trial & cons & trial & cons & interact & nobs \\
\hline
\\
Better off than average of village?
& 0.35
& 0.02$^{}$
& 0.00$^{}$ 
& 0.00$^{}$ 
& -0.02$^{}$
& 0.05$^{}$ & 1504 \\

& (0.48) & (0.03)  & (0.03) & (0.05)  & (0.05)  & (0.07)
 \\

Better off than 6 months ago?
& 0.42
& 0.12$^{**}$
& -0.02$^{}$ 
& 0.10$^{+}$ 
& -0.05$^{}$
& 0.06$^{}$ & 1531 \\

& (0.49) & (0.03)  & (0.03) & (0.05)  & (0.05)  & (0.07)
 \\

Can always eat what they want?
& 0.49
& 0.03$^{}$
& 0.03$^{}$ 
& -0.07$^{}$ 
& -0.07$^{}$
& 0.21$^{*}$ & 1538 \\

& (0.50) & (0.05)  & (0.05) & (0.07)  & (0.07)  & (0.10)
 \\

Can always eat quantity needed?
& 0.59
& 0.08$^{+}$
& -0.04$^{}$ 
& 0.01$^{}$ 
& -0.10$^{+}$
& 0.13$^{}$ & 1538 \\

& (0.49) & (0.04)  & (0.04) & (0.05)  & (0.06)  & (0.08)
 \\

Consumption expenditure (*1000 UGX/week)
& 86145
& 84$^{}$
& -3175$^{}$ 
& 1875$^{}$ 
& -1392$^{}$
& -3585$^{}$ & 1538 \\

& (46538) & (4062)  & (4061) & (5768)  & (5717)  & (8123)
 \\
\\
Index
& 0.00
& 0.08$^{}$
& -0.02$^{}$ 
& 0.01$^{}$ 
& -0.09$^{}$
& 0.14$^{}$ & 1502 \\

& (0.61) & (0.05)  & (0.05) & (0.07)  & (0.07)  & (0.10)
 \\
\\
\hline\hline
\end{tabular}
\end{center}
\scriptsize
Note: Column (1) reports sample means at baseline and standard deviations below;  columns (2)-(3) report differences between treatment and control groups estimated using the pooled model of Equation  \ref{eq:pooled}; (4)-(5) report differences between treatment and control groups estimated using the fully interacted model of Equation \ref{eq:one}; column (6) is the interaction effect. Standared errors are below the estimates and are clustered at the level of randomization; **, *, and + denote significance at the 1, 5 and 10\% levels.
\end{sidewaystable}


\section{Results: Mechanisms}

\subsection{Knowledge, risk, social learning, and intentions}

To explore potential impact pathways, we posed several questions.
First, we aimed to assess whether the interventions increased awareness
of improved maize seeds, specifically the Bazooka variety. We asked
farmers to list as many improved maize seed varieties as they could
name, and enumerators recorded the number of varieties mentioned.
Second, we directly inquired whether the farmer was familiar with
a maize seed variety called “Bazooka.”

Table \ref{pathways} shows a clear impact on seed knowledge resulting
from the two interventions. On average, farmers are aware of approximately
2 or 3 different maize seed varieties. Both interventions lead to
a substantial increase in the number of seed varieties known by farmers,
with the effects being quite similar. Overall, 82 percent of farmers
in our sample are familiar with the Bazooka variety. The intervention
with the most pronounced impact on this specific knowledge was the
seed trial pack, likely because it directly involved the Bazooka variety,
with its name prominently displayed on the packaging. In contrast,
the cooking and tasting demonstration had a more general focus on
the consumption traits of improved seed varieties, which may explain
its comparatively lower impact on specific seed recognition.\footnote{The more general focus was deliberate as we did not want to be seen
as promoting a particular commercial variety. However, we did mention
that the maize flour that we used (and also provided a take home sample
from) came from Bazooka. This is why the result for the cooking demonstration
is small but still significant (at least in the pooled model), and
also why we do see that farmers in demo group also adopted more fresh
bazooka (see Tables \ref{adoptiom} and \ref{adoptiom-1})}

As noted in the introduction, seed trial packs aim to mitigate the
perceived risk associated with trying new seed varieties. To assess
changes in perceived risk in response to the interventions, we asked
farmers to rate how likely they believe it is that using improved
seed varieties would result in lower yields compared to local seed.
Responses were recorded on a 4-point Likert scale, ranging from \textquotedbl very
likely\textquotedbl{} (improved seed will yield less than local) to
\textquotedbl very unlikely\textquotedbl{} (improved seed will yield
more than local). We repeated this question specifically for the seed
type used in the study, but only for farmers who reported familiarity
with the seed. We constructed an indicator for downside risk, which
is set to 1 if the farmer responded with a 1 or 2 on the Likert scale
(indicating \textquotedbl very likely\textquotedbl{} or \textquotedbl somewhat
likely\textquotedbl{} that improved seed will yield less). Our findings
reveal that perceived downside risk is generally limited, and neither
intervention significantly affected risk perceptions related to improved
seed in general nor to the specific seed used in the study.

An additional rationale for subsidizing or providing seed at no cost
is to harness potential spillover effects to enhance adoption. To
examine whether our interventions influenced social learning, we included
questions on whether farmers recommended any improved seed varieties
to others. A similar question was posed specifically regarding the
seed variety used in the trial pack, but only for those farmers who
were familiar with Bazooka.

Our findings indicate that a substantial proportion of farmers recommend
improved seed to their peers. Notably, the likelihood of recommending
improved seed is significantly higher among those who received the
seed trial pack. When focusing specifically on Bazooka, the difference
becomes even more pronounced. Receiving a seed trial pack increases
the probability of recommending Bazooka by 38 to 44 percentage points,
depending on the model used.

Finally, we examined farmers' intentions regarding future use of improved
seed. Although a considerable proportion of farmers expressed an intention
to use improved seed in the future, we did not observe a significant
impact from either the seed trial pack nor the cooking demonstration
and tasting session on these intentions. Furthermore, we find that,
overall, about 36 
percent of farmers indicate that they are very likely to use bazooka
in the next season. We see that in the group that received a seed
trial pack, the percentage is 16
to 20 percentage
points higher than in the group that did not receive a seed trial
pack. When summarizing these outcomes into an index, we observe a
positive impact from the seed trial pack. In contrast, the effect
of the cooking demonstration and tasting session is less convincing.

\begin{sidewaystable}
\caption{Knowledge, risk, social learning, and intentions\label{pathways}}

\footnotesize
\begin{center}
\begin{tabular}{rccccccc}
\hline\hline
& & \multicolumn{2}{c}{\textit{Pooled model}} & \multicolumn{3}{c}{\textit{Interacted model}} \\
& mean & trial & cons & trial & cons & interact & nobs \\
\hline
\\
Knows bazooka (yes=1)
& 0.82
& 0.26$^{**}$
& 0.06$^{*}$ 
& 0.29$^{**}$ 
& 0.08$^{}$
& 0.17$^{**}$ & 1538 \\

& (0.38) & (0.03)  & (0.03) & (0.04)  & (0.05)  & (0.02)
 \\

Number of improved seed farmer knows
& 2.66
& 0.40$^{**}$
& 0.42$^{**}$ 
& 0.48$^{**}$ 
& 0.50$^{**}$
& -0.16$^{}$ & 1532 \\

& (1.50) & (0.12)  & (0.12) & (0.16)  & (0.16)  & (0.24)
 \\

Thinks improved seed is risky
& 0.09
& 0.00$^{}$
& 0.00$^{}$ 
& 0.00$^{}$ 
& 0.01$^{}$
& -0.01$^{}$ & 1447 \\

& (0.28) & (0.02)  & (0.02) & (0.03)  & (0.03)  & (0.04)
 \\

Thinks bazooka is risky
& 0.13
& 0.01$^{}$
& -0.02$^{}$ 
& 0.03$^{}$ 
& 0.01$^{}$
& -0.05$^{}$ & 1207 \\

& (0.34) & (0.04)  & (0.04) & (0.05)  & (0.06)  & (0.08)
 \\

Recommended improved seed to others
& 0.59
& 0.29$^{**}$
& 0.02$^{}$ 
& 0.29$^{**}$ 
& 0.02$^{}$
& 0.01$^{}$ & 1538 \\

& (0.49) & (0.04)  & (0.04) & (0.06)  & (0.05)  & (0.07)
 \\

Recommended bazooka to other
& 0.62
& 0.38$^{**}$
& 0.06$^{}$ 
& 0.44$^{**}$ 
& 0.11$^{+}$
& -0.11$^{}$ & 1260 \\

& (0.48) & (0.04)  & (0.04) & (0.06)  & (0.07)  & (0.08)
 \\

Will use improved seed in the future
& 0.83
& 0.05$^{}$
& 0.01$^{}$ 
& 0.05$^{}$ 
& 0.01$^{}$
& 0.00$^{}$ & 1461 \\

& (0.38) & (0.03)  & (0.03) & (0.05)  & (0.05)  & (0.07)
 \\

Will use bazooka in the future
& 0.36
& 0.16$^{**}$
& 0.02$^{}$ 
& 0.20$^{**}$ 
& 0.05$^{}$
& -0.07$^{}$ & 1503 \\

& (0.48) & (0.04)  & (0.04) & (0.06)  & (0.05)  & (0.09)
 \\
\\

Index
& 0.38
& 0.14$^{**}$
& 0.05$^{+}$ 
& 0.16$^{**}$ 
& 0.07$^{}$
& -0.03$^{}$ & 1156 \\

& (0.36) & (0.03)  & (0.03) & (0.05)  & (0.05)  & (0.06)
 \\
\\
\hline\hline
\end{tabular}
\end{center}
\scriptsize
Note: Column (1) reports sample means at baseline and standard deviations below;  columns (2)-(3) report differences between treatment and control groups estimated using the pooled model of Equation  \ref{eq:pooled}; (4)-(5) report differences between treatment and control groups estimated using the fully interacted model of Equation \ref{eq:one}; column (6) is the interaction effect. Standared errors are below the estimates and are clustered at the level of randomization; **, *, and + denote significance at the 1, 5 and 10\% levels.
\end{sidewaystable}


\subsection{Perceptions}

\subsubsection{Perceptions of consumer traits}

Farmers may perceive that maize grown from local seed is tastier than
maize from commercial varieties \citep{picha2018preference}. The
cooking demonstrations and tasting sessions are intended to alter
these perceptions about the consumption qualities of maize from improved
seed varieties. Additionally, if farmers plant the seed trial pack
and process and cook the harvest separately, they may revise their
preconceived beliefs about the consumer traits of these improved varieties.
In Table \ref{cons_traits_used}, we examine the effects of the two
interventions on perceptions of consumer traits related to the harvest
from the seed used in the previous season. Specifically, we assessed
whether farmers found the taste of the maize to be better than expected,
whether the yield (i.e., the amount of maize that can be prepared
from a given quantity) was more satisfactory, whether the appearance
was improved (often indicating a whiter posho), and whether the ease
of cooking was better.

Results show that 72
percent of farmers indicate that the maize flour obtained from the
randomly selected plot tasted better or much better than what they
expected expected. A slightly lower share of farmers, 65
percent, indicated that portions obtained from the maize turn out
larger (or much larger than expected). A similar proportion of farmers
reported that the appearance of the maize, typically indicated by
its whiteness, was either better or much better than expected. Additionally,
just over half of the farmers found that the maize grain from the
randomly selected plot was (much) easier to cook than they had anticipated.

We observe that the seed trial intervention resulted in significant
improvements across all these measures. The treatment effects are
approximately 20 percentage points, indicating a substantial impact.
In contrast, we find no significant effect from the cooking demonstration
and tasting session.

\begin{sidewaystable}
\caption{Impact on Consumption traits of seed used\label{cons_traits_used}}

\footnotesize
\begin{center}
\begin{tabular}{rccccccc}
\hline\hline
& & \multicolumn{2}{c}{\textit{Pooled model}} & \multicolumn{3}{c}{\textit{Interacted model}} \\
& mean & trial & cons & trial & cons & interact & nobs \\
\hline
\\
Taste
& 0.72
& 0.17$^{**}$
& 0.01$^{}$ 
& 0.19$^{**}$ 
& 0.03$^{}$
& -0.04$^{}$ & 1452 \\

& (0.45) & (0.04)  & (0.04) & (0.05)  & (0.06)  & (0.07)
 \\

Portions
& 0.65
& 0.19$^{**}$
& -0.01$^{}$ 
& 0.18$^{**}$ 
& -0.02$^{}$
& 0.02$^{}$ & 1400 \\

& (0.48) & (0.04)  & (0.04) & (0.06)  & (0.05)  & (0.08)
 \\

Appearance
& 0.65
& 0.24$^{**}$
& 0.05$^{}$ 
& 0.25$^{**}$ 
& 0.06$^{}$
& -0.02$^{}$ & 1447 \\

& (0.48) & (0.04)  & (0.04) & (0.06)  & (0.05)  & (0.07)
 \\

Ease of cooking
& 0.56
& 0.22$^{**}$
& 0.03$^{}$ 
& 0.24$^{**}$ 
& 0.05$^{}$
& -0.04$^{}$ & 1387 \\

& (0.50) & (0.04)  & (0.04) & (0.06)  & (0.06)  & (0.09)
 \\
\\
Index
& 0.24
& 0.41$^{**}$
& 0.03$^{}$ 
& 0.43$^{**}$ 
& 0.05$^{}$
& -0.04$^{}$ & 1345 \\

& (0.75) & (0.07)  & (0.07) & (0.10)  & (0.10)  & (0.13)
 \\

\\
\hline\hline
\end{tabular}
\end{center}
\scriptsize
Note: Column (1) reports sample means at baseline and standard deviations below;  columns (2)-(3) report differences between treatment and control groups estimated using the pooled model of Equation  \ref{eq:pooled}; (4)-(5) report differences between treatment and control groups estimated using the fully interacted model of Equation \ref{eq:one}; column (6) is the interaction effect. Standared errors are below the estimates and are clustered at the level of randomization; **, *, and + denote significance at the 1, 5 and 10\% levels.
\end{sidewaystable}

The results in Table \ref{cons_traits_used} are for the seed that
was used by the farmer, only some of which may be using improved seed
varieties. Therefore, we also included a module in the questionnaire
asking farmers to compare maize from improved seed varieties (such
as Longe5 or Bazooka) with maize from local seed across the same four
consumption traits. In other words, this module directly compares
the consumption attributes of improved seed with those of local seed.

Results are summarized in Table \ref{cons_traits_compare}. As in
Table \ref{cons_traits_used}, the overall satisfaction with the consumption
traits of improved seed is high across the sample. Furthermore, the
seed trial pack intervention significantly increased the proportion
of farmers who perceive improved seed as superior to local seed across
various consumption attributes. Notably, we now also observe that
the cooking demonstration and tasting session has influenced consumer
perceptions, demonstrating its effectiveness in altering views on
the quality of improved seed.

\begin{sidewaystable}
\caption{Impact on Consumption traits - improved seed compared to local\label{cons_traits_compare}}

\footnotesize
\begin{center}
\begin{tabular}{rccccccc}
\hline\hline
& & \multicolumn{2}{c}{\textit{Pooled model}} & \multicolumn{3}{c}{\textit{Interacted model}} \\
& mean & trial & cons & trial & cons & interact & nobs \\
\hline
\\
Taste
& 0.70
& 0.21$^{**}$
& 0.09$^{*}$ 
& 0.27$^{**}$ 
& 0.15$^{*}$
& -0.11$^{}$ & 1424 \\

& (0.46) & (0.04)  & (0.04) & (0.06)  & (0.06)  & (0.08)
 \\

Portions
& 0.81
& 0.10$^{**}$
& 0.10$^{**}$ 
& 0.14$^{**}$ 
& 0.15$^{**}$
& -0.08$^{}$ & 1360 \\

& (0.40) & (0.03)  & (0.03) & (0.04)  & (0.05)  & (0.06)
 \\

Appearance
& 0.85
& 0.05$^{+}$
& 0.09$^{**}$ 
& 0.06$^{}$ 
& 0.10$^{*}$
& -0.02$^{}$ & 1421 \\

& (0.36) & (0.03)  & (0.03) & (0.05)  & (0.04)  & (0.05)
 \\

Ease of cooking
& 0.71
& 0.14$^{**}$
& 0.13$^{**}$ 
& 0.19$^{**}$ 
& 0.18$^{**}$
& -0.10$^{}$ & 1316 \\

& (0.45) & (0.04)  & (0.04) & (0.06)  & (0.06)  & (0.08)
 \\
\\
Index
& 0.32
& 0.23$^{**}$
& 0.20$^{**}$ 
& 0.29$^{**}$ 
& 0.26$^{**}$
& -0.12$^{}$ & 1248 \\

& (0.60) & (0.05)  & (0.05) & (0.09)  & (0.09)  & (0.11)
 \\
\\

\hline\hline
\end{tabular}
\end{center}
\scriptsize
Note: Column (1) reports sample means at baseline and standard deviations below;  columns (2)-(3) report differences between treatment and control groups estimated using the pooled model of Equation  \ref{eq:pooled}; (4)-(5) report differences between treatment and control groups estimated using the fully interacted model of Equation \ref{eq:one}; column (6) is the interaction effect. Standared errors are below the estimates and are clustered at the level of randomization; **, *, and + denote significance at the 1, 5 and 10\% levels.
\end{sidewaystable}

The differences between Table \ref{cons_traits_compare} and \ref{cons_traits_used}
align with the findings reported in Tables \ref{adoptiom} and \ref{adoptiom-1}.
Specifically, the negative impact on adoption observed with the seed
trial pack can be attributed to a significant proportion of farmers
recycling the trial seed. These farmers are more likely to report
that the grain from the trial seed exceeded their expectations. Conversely,
farmers in the cooking demonstration group increased their adoption
of bazooka seed. Although this effect is statistically significant,
it is economically modest, with an increase of only five percentage
points in adoption. These farmers primarily grew local seed, resulting
in consumption traits aligning with their expectations. However, when
asked to compare consumption traits of improved seed varieties, such
as bazooka, to local seed, the cooking demonstration and tasting session
effectively altered perceptions, as expected.

\subsubsection{Perceptions of producer traits}

The seed trial pack is also designed to address preconceived notions
farmers may have about the production-related traits of improved seed.
For example, qualitative fieldwork conducted during the study’s preparation
revealed that some farmers believed improved seed varieties might
offer higher yields but were less resistant to fall armyworm (\emph{Spodoptera
frugiperda}) infestations. Others, having been disappointed in the
past, might no longer believe in the yield advantages of improved
seeds \citep{miehe2023miracle}. Providing free seed trial packs can
be an effective way to alter these perceptions, as it allows farmers
to directly experience the production traits of new technology on
their own fields \citep{doi:10.1086/601447}.

In Table \ref{prod_traits_used}, we therefore examine the effects
of the two interventions on perceptions regarding the production-related
characteristics of the seed used by farmers in the previous season.
Specifically, we assess whether farmers perceived the following aspects
as exceeding their expectations: yield, tolerance to abiotic stresses
(such as drought and heat), tolerance to biotic stresses (such as
pests, diseases, and weeds), germination rates, and time to maturity.

\begin{sidewaystable}
\caption{Impact on Production traits of seed used\label{prod_traits_used}}

\footnotesize
\begin{center}
\begin{tabular}{rccccccc}
\hline\hline
& & \multicolumn{2}{c}{\textit{Pooled model}} & \multicolumn{3}{c}{\textit{Interacted model}} \\
& mean & trial & cons & trial & cons & interact & nobs \\
\hline
\\
Yield
& 0.24
& 0.14$^{**}$
& 0.05$^{}$ 
& 0.13$^{**}$ 
& 0.04$^{}$
& 0.02$^{}$ & 1462 \\

& (0.43) & (0.03)  & (0.03) & (0.04)  & (0.03)  & (0.06)
 \\

Abiotic stresses
& 0.43
& 0.11$^{*}$
& 0.00$^{}$ 
& 0.15$^{*}$ 
& 0.04$^{}$
& -0.07$^{}$ & 968 \\

& (0.49) & (0.04)  & (0.04) & (0.07)  & (0.05)  & (0.09)
 \\

Biotic stresses
& 0.39
& 0.17$^{**}$
& 0.03$^{}$ 
& 0.24$^{**}$ 
& 0.10$^{*}$
& -0.14$^{*}$ & 1381 \\

& (0.49) & (0.03)  & (0.03) & (0.05)  & (0.04)  & (0.07)
 \\

Time to maturity
& 0.65
& 0.26$^{**}$
& 0.02$^{}$ 
& 0.28$^{**}$ 
& 0.04$^{}$
& -0.05$^{}$ & 1446 \\

& (0.48) & (0.03)  & (0.03) & (0.05)  & (0.05)  & (0.07)
 \\

Germination Rate
& 0.70
& 0.17$^{**}$
& 0.03$^{}$ 
& 0.21$^{**}$ 
& 0.06$^{}$
& -0.07$^{}$ & 1465 \\

& (0.46) & (0.04)  & (0.04) & (0.06)  & (0.06)  & (0.08)
 \\
\\
Index
& 0.25
& 0.39$^{**}$
& 0.07$^{}$ 
& 0.45$^{**}$ 
& 0.13$^{}$
& -0.12$^{}$ & 941 \\

& (0.70) & (0.07)  & (0.07) & (0.11)  & (0.08)  & (0.13)
 \\
\\
\hline\hline
\end{tabular}
\end{center}
\scriptsize
Note: Column (1) reports sample means at baseline and standard deviations below;  columns (2)-(3) report differences between treatment and control groups estimated using the pooled model of Equation  \ref{eq:pooled}; (4)-(5) report differences between treatment and control groups estimated using the fully interacted model of Equation \ref{eq:one}; column (6) is the interaction effect. Standared errors are below the estimates and are clustered at the level of randomization; **, *, and + denote significance at the 1, 5 and 10\% levels.
\end{sidewaystable}

Overall, farmers appear to be more critical regarding the production
traits of the seed used on the randomly selected plot. For instance,
only approximately 24
percent of farmers reported that the yield from the seed on the randomly
selected plot was (much) higher than what they had anticipated.

As with consumer traits, the seed trial pack significantly impacts
all production-related traits. Specifically, a notably higher proportion
of farmers who received the seed trial pack report that the yield
exceeded their expectations compared to those who did not receive
the trial pack. Conversely, we do not observe any significant impact
from the cooking demonstration and tasting session on production traits,
which aligns with our expectations.

As for consumption traits above, the results in Table \ref{prod_traits_used}
are for the seed that was used by the farmer, only some of which may
be using improved seed varieties. Therefore, we also include a module
in the questionnaire where we ask farmers to compare seed of an improved
variety to local seed on the same five production traits. Results
are summarized in Table \ref{cons_traits_compare}.

\begin{sidewaystable}
\caption{Impact on Production traits - improved seed compared to local\label{prod_traits_compared}}

\footnotesize
\begin{center}
\begin{tabular}{rccccccc}
\hline\hline
& & \multicolumn{2}{c}{\textit{Pooled model}} & \multicolumn{3}{c}{\textit{Interacted model}} \\
& mean & trial & cons & trial & cons & interact & nobs \\
\hline
\\
Yield
& 0.92
& 0.03$^{}$
& 0.00$^{}$ 
& 0.03$^{}$ 
& 0.01$^{}$
& -0.02$^{}$ & 1464 \\

& (0.27) & (0.02)  & (0.02) & (0.03)  & (0.03)  & (0.04)
 \\

Abiotic stresses
& 0.72
& 0.06$^{+}$
& -0.03$^{}$ 
& 0.10$^{+}$ 
& 0.00$^{}$
& -0.06$^{}$ & 1336 \\

& (0.45) & (0.04)  & (0.04) & (0.05)  & (0.06)  & (0.08)
 \\

Biotic stresses
& 0.58
& 0.09$^{*}$
& -0.01$^{}$ 
& 0.12$^{+}$ 
& 0.02$^{}$
& -0.05$^{}$ & 1391 \\

& (0.49) & (0.04)  & (0.04) & (0.06)  & (0.06)  & (0.09)
 \\

Time to maturity
& 0.93
& 0.01$^{}$
& 0.00$^{}$ 
& -0.01$^{}$ 
& -0.02$^{}$
& 0.05$^{}$ & 1446 \\

& (0.25) & (0.02)  & (0.02) & (0.03)  & (0.02)  & (0.04)
 \\

Germination Rate
& 0.84
& 0.08$^{**}$
& 0.05$^{+}$ 
& 0.09$^{+}$ 
& 0.06$^{}$
& -0.02$^{}$ & 1439 \\

& (0.36) & (0.03)  & (0.03) & (0.05)  & (0.04)  & (0.06)
 \\
\\
Index
& 0.10
& 0.11$^{*}$
& 0.03$^{}$ 
& 0.13$^{+}$ 
& 0.05$^{}$
& -0.04$^{}$ & 1273 \\

& (0.58) & (0.05)  & (0.05) & (0.08)  & (0.07)  & (0.10)
 \\

\hline\hline
\end{tabular}
\end{center}
\scriptsize
Note: Column (1) reports sample means at baseline and standard deviations below;  columns (2)-(3) report differences between treatment and control groups estimated using the pooled model of Equation  \ref{eq:pooled}; (4)-(5) report differences between treatment and control groups estimated using the fully interacted model of Equation \ref{eq:one}; column (6) is the interaction effect. Standared errors are below the estimates and are clustered at the level of randomization; **, *, and + denote significance at the 1, 5 and 10\% levels.
\end{sidewaystable}

Overall averages indicate that most farmers perceive improved seed
varieties to be superior in several production traits. For example,
92
percent of farmers acknowledge that improved seed yields more or much
more than local seed varieties. Similarly, a high proportion of farmers
appreciate the shorter time to maturity and better germination rates
of improved seed. However, farmers are somewhat less convinced about
the improved seed's resistance to biotic stressors. This skepticism
aligns with the perception that improved seed may be less resistant
to fall army-worm infestation, a significant issue in the region.

When farmers compare improved seed directly to local seed, the summary
index indicates a significantly positive impact from the seed trial
pack. Farmers who received the seed trial pack updated their beliefs
most notably regarding germination rates. Interestingly, these farmers
also revised their previously pessimistic views on biotic stress resistance.
Although the effects on other attributes are positive, the statistical
power may be limited due to already high baseline perceptions of improved
seed varieties. No significant impact was observed from the cooking
demonstration and tasting session, which aligns with our expectations.

\subsubsection{Perceptions on post-harvest traits}

We also examine perceptions of post-harvest traits that could influence
adoption decisions and might be affected by our interventions. Specifically,
we assess perceptions related to marketability (e.g., whether the
seed fetches a high price, is easy to sell), biomass or crop residues
(e.g., maize stalks and maize bran for use as organic fertilizer or
animal feed), and processing ease (e.g., shelling, milling). These
traits are considered separately as they straddle the line between
production and consumer traits. They relate to the utilization of
the product obtained from the seed, but we anticipate an impact primarily
from the intervention targeting production-related traits.

\begin{sidewaystable}
\caption{Impact on post harvest traits of seed used\label{ph_traits_used}}

\footnotesize
\begin{center}
\begin{tabular}{rccccccc}
\hline\hline
& & \multicolumn{2}{c}{\textit{Pooled model}} & \multicolumn{3}{c}{\textit{Interacted model}} \\
& mean & trial & cons & trial & cons & interact & nobs \\
\hline
\\
Marketability
& 0.55
& 0.12$^{*}$
& 0.01$^{}$ 
& 0.14$^{+}$ 
& 0.03$^{}$
& -0.05$^{}$ & 1021 \\

& (0.50) & (0.06)  & (0.06) & (0.08)  & (0.07)  & (0.11)
 \\

Biomass
& 0.33
& 0.00$^{}$
& 0.08$^{*}$ 
& 0.02$^{}$ 
& 0.10$^{*}$
& -0.05$^{}$ & 1395 \\

& (0.47) & (0.03)  & (0.03) & (0.05)  & (0.04)  & (0.07)
 \\

Easy to process
& 0.55
& 0.13$^{**}$
& -0.02$^{}$ 
& 0.10$^{}$ 
& -0.05$^{}$
& 0.07$^{}$ & 1245 \\

& (0.50) & (0.05)  & (0.05) & (0.07)  & (0.07)  & (0.10)
 \\
\\
Index
& 0.14
& 0.15$^{+}$
& 0.02$^{}$ 
& 0.19$^{}$ 
& 0.07$^{}$
& -0.09$^{}$ & 856 \\

& (0.69) & (0.08)  & (0.08) & (0.12)  & (0.12)  & (0.16)
 \\

\\
\hline\hline
\end{tabular}
\end{center}
\scriptsize
Note: Column (1) reports sample means at baseline and standard deviations below;  columns (2)-(3) report differences between treatment and control groups estimated using the pooled model of Equation  \ref{eq:pooled}; (4)-(5) report differences between treatment and control groups estimated using the fully interacted model of Equation \ref{eq:one}; column (6) is the interaction effect. Standared errors are below the estimates and are clustered at the level of randomization; **, *, and + denote significance at the 1, 5 and 10\% levels.
\end{sidewaystable}

Table \ref{ph_traits_used} shows results for perceptions related
to post harvest characteristics of the seed used in the randomly selected
plot, while Table \ref{ph_traits_compare} presents direct comparison
between improved seed varieties and local seed. We find weak evidence
that farmers in the group that received the seed trail pack rate the
seed they used on the randomly selected plot higher on a range of
post harvest traits than those that did not get a seed trial pack.
We come to a similar conclusion when farmers are asked to directly
compare improved seed to local seed. As expected, we find no impact
of the cooking demonstration and tasting session.

\begin{sidewaystable}
\caption{Impact on post harvest traits - improved seed compared to local\label{ph_traits_compare}}

\footnotesize
\begin{center}
\begin{tabular}{rccccccc}
\hline\hline
& & \multicolumn{2}{c}{\textit{Pooled model}} & \multicolumn{3}{c}{\textit{Interacted model}} \\
& mean & trial & cons & trial & cons & interact & nobs \\
\hline
\\
Marketability
& 0.52
& 0.09$^{+}$
& 0.04$^{}$ 
& 0.15$^{*}$ 
& 0.10$^{}$
& -0.11$^{}$ & 1344 \\

& (0.50) & (0.05)  & (0.05) & (0.06)  & (0.07)  & (0.09)
 \\

Biomass
& 0.37
& -0.01$^{}$
& 0.00$^{}$ 
& 0.02$^{}$ 
& 0.03$^{}$
& -0.05$^{}$ & 1351 \\

& (0.48) & (0.04)  & (0.04) & (0.06)  & (0.06)  & (0.08)
 \\

Easy to process
& 0.70
& 0.06$^{}$
& -0.01$^{}$ 
& 0.08$^{}$ 
& 0.02$^{}$
& -0.04$^{}$ & 1236 \\

& (0.46) & (0.05)  & (0.05) & (0.07)  & (0.07)  & (0.09)
 \\
\\
Index
& 0.12
& 0.09$^{}$
& 0.00$^{}$ 
& 0.17$^{+}$ 
& 0.08$^{}$
& -0.16$^{}$ & 1097 \\

& (0.68) & (0.07)  & (0.07) & (0.10)  & (0.11)  & (0.14)
 \\
\\

\hline\hline
\end{tabular}
\end{center}
\scriptsize
Note: Column (1) reports sample means at baseline and standard deviations below;  columns (2)-(3) report differences between treatment and control groups estimated using the pooled model of Equation  \ref{eq:pooled}; (4)-(5) report differences between treatment and control groups estimated using the fully interacted model of Equation \ref{eq:one}; column (6) is the interaction effect. Standared errors are below the estimates and are clustered at the level of randomization; **, *, and + denote significance at the 1, 5 and 10\% levels.
\end{sidewaystable}


\section{Conclusion}

In this study, we conduct a field experiment to evaluate the effectiveness
of two interventions aimed at increasing the adoption of improved
maize seed varieties among smallholder farmers in eastern Uganda.
The first intervention provides free seed trial packs, allowing farmers
to assess the production-related attributes of the new variety, including
its yield potential, pest tolerance, and germination rate. The second
intervention consists of a cooking demonstration and tasting session
designed to emphasize the consumption-related attributes of the new
variety, such as taste, texture, and color of the food produced from
the new seed.

The interventions are tested using a cluster randomized controlled
trial involving 1,560 maize farmers. The trial is clustered at the
village level, with 10 households randomly selected from each village,
and villages are assigned to treatments according to a 2x2 factorial
design.

Our findings indicate that the seed trial pack intervention has an
unintended effect: it reduces the likelihood of adopting improved
seed in the subsequent season. This counterintuitive result arises
because farmers who received the trial pack are more inclined to recycle
the seed from the trial pack, rather than purchase fresh seed. Notably,
recycled seed from the trial pack is not categorized as “improved
seed” in our analysis.

Despite the unexpected outcome concerning seed adoption, the intermediary
variables align with our expectations and indicate that the interventions
achieved their intended effects. Specifically, the seed trial pack
improves perceptions of production-related traits, while the cooking
demonstration and tasting session enhance perceptions of consumption
traits.

Interestingly, the seed trial pack intervention also positively influences
perceptions of consumption-related traits, in addition to production
traits. From a cost-efficiency standpoint, and assuming comparable
costs for both interventions, the seed trial pack appears to be a
more effective approach.

The main lesson from the seed trial pack experiment is the necessity
of addressing additional constraints. While the trial pack positively
impacts farmer perceptions and intentions, the observed preference
for recycling old seed over purchasing fresh seed suggests issues
related to seed access. Potential barriers include the unavailability
of improved seed at local agro-input shops or high seed prices. Ultimately,
addressing these access issues alongside intervention strategies will
be crucial for maximizing the impact of improved seed adoption and
enhancing overall agricultural productivity in the region.

\section{Acknowledgments}

This research received clearance form Makerere's School of Social
Sciences Research Ethics Committee (MAKSSREC 01.23.627/PR1) as well
as from IFPRI IRB (DSGD-23-0108). The research was also registered
at the Ugandan National Commission for Science and Technology (SS1657ES).
This research is part of the \href{https://www.cgiar.org/initiative/market-intelligence/}{OneCG Market Intelligence Initiative}
which is funded by a \href{https://www.cgiar.org/funders/}{consortium of donors}.
We want to thank Richard Ariong, Wilberfoce Walukano and Marc Charles
Wanume for field support.

\bibliographystyle{2_home_bjvca_data_projects_OneCG_MIPP_study_design_IFPRI_Style}
\bibliography{1_home_bjvca_data_projects_OneCG_MIPP_papers_increasing_seed_varietal_turnover_MIPPI_PAP}

\end{document}
